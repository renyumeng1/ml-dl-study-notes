% 纸张模式:护眼模式(geye)和朦胧模式(hazy);
% 适配不同设备,包括 Pad(默认),Screen(幻灯片),Kindle,PC(双页),通用(A4 纸张);
% 5 套颜色主题,分别是:\textcolor{eblue}{blue}(默认)、\textcolor{egreen}{green}、\textcolor{ecyan}{cyan}、 \textcolor{sakura}{sakura} 和 \textcolor{black}{black};
% 语言支持:中文(默认),英文;
% 支持 \hologo{pdfLaTeX} 和 \hologo{XeLaTeX} 编译;
% 更加美观的图表标题格式,列表环境,数学字体等;
% 全局字体大小支持:8pt, 9pt, 10pt, 11pt, 12pt, 14pt, 17pt, 和 20pt;
% 支持 \lstinline{newtx} 以及 \lstinline{mtpro2} 数学字体设置; 
% 中文字体支持方正字体或者自定义字体;
% 英文模式通过 \lstinline{bibstyle} 选项(默认为 \lstinline{apalike})支持参考文献格式修改;
% 支持参考格式显示格式修改 \lstinline{cite} 可选为 \lstinline{authoryear}、\lstinline{numbers} (默认)和 \lstinline{super}。
% \documentclass[device=pad]{elegantnote}    % ipad screen size
% \documentclass[device=kindle]{elegantnote} % kindle screen size
% \documentclass[device=pc]{elegantnote}     % double pages for pc 
% \documentclass[device=normal]{elegantnote} % a4 normal page
% \documentclass[device=screen]{elegantnote} % 4:3 PPT size
%!TEX program = xelatex
\documentclass[cn,hazy,blue,14pt,normal]{elegantnote}

\title{深度学习个人笔记}
\author{Amamiyaren}
\institute{my own deep learning notes}
\date{\zhdate{2024/5/17}}
\usepackage{array}

\begin{document}
\maketitle
\numberwithin{equation}{section}
\section{多元线性回归}

\subsection{基本形式}
线性模型试图学得一个通过属性的线性组合来预测的函数,基本形式如下:
\begin{align}
    f(\textbf{x})=w_1x_1+w_2x_2+\dots+w_dx_d+b
\end{align}
向量形式如下:
\begin{align}
    f(\textbf{x})=\textbf{w}^{\textbf{T}}\textbf{x}+\textbf{b}
\end{align}
\subsection{一元情形}
一般采用最小二乘法确定模型参数,一元情形推导如下:
\begin{align}
  (w^*,b^*) &= \mathop{\arg\min}\limits_{(w,b)}\sum_{i=1}^{m}{(f(x_i)-yi)^2}\nonumber\\&=\mathop{\arg\min}\limits_{(w,b)}\sum_{i=1}^{m}{(y_i-wx_i-b)^2}
 \end{align}
 我们令$E_{(w,b)}=\mathop{\arg\min}\limits_{(w,b)}\sum_{i=1}^{m}{(y_i-wx_i-b)^2}$,等式两边同时对$w$和$b$求导有:
 
 \begin{align}
  \frac{\partial E_{(w,b)}}{\partial b} &=2\Big(mb- \sum_{i=1}^m(y_i-wx_i)\Big)
\end{align}

\begin{align}
  \frac{\partial E_{(w,b)}}{\partial w} &= \sum_{i=1}^{m}2(y_i-wx_i-b)(-x_i) \nonumber \\ &= 2 \Big (w\sum_{i_1}^mx^2-\sum_{i=1}^m(y_i-b)x_i\Big )
\end{align}
然后令式(1.4)和(1.5)为零得到:
\begin{align}
  b&=\frac{1}{m}\sum_{i=1}^my_i-wx_i \nonumber\\&=\bar{y}-w\bar{x}\
\end{align}

\begin{align}
  w\sum_{i=1}^m{x_i^2}=\sum_{i=1}^m{(y_i-b)x_i}
\end{align}

将(1.6)带入(1.7)得到:
\begin{align}
  w\sum_{i=1}^m{x_i^2}&=\sum_{i=1}^m{(y_ix_i)-\sum_{i=1}^m(\bar{y}-w\bar{x})x_i}\nonumber \\ &=\sum_{i=1}^my_ix_i - \bar{y}\sum_{i=1}^mx_i+w\bar{x}\sum_{i=1}^mx_i
\end{align}
对(1.8)式进行化简,最终可得:
\begin{align}
  w = \frac{\sum_{i=1}^my_ix_i-\bar{y}\sum_{i=1}^mx_i}{\sum_{i=1}^mx_i^2-\sum_{i=1}^m\bar{x}x_i}
\end{align}

又由于$\bar{y}\sum_{i=1}^mx_i=\frac{1}{m}\sum_{i=1}^m{y_i}\cdot\sum_{i=1}^mx_i=\bar{x}\sum_{i=1}^my_i$和$\bar{x}\sum_{i=1}^my_i=\frac{1}{m}\sum_{i=1}^mx_i\sum_{i=1}^mx_i=\frac{1}{m}\Big(\sum_{i=1}^mx_i\Big)^2$将这两个式子带入(1.9)式进行进一步化简得到:
\begin{align}
  w=\frac{\sum_{i=1}^my_i(x_i-\bar{x})}{\sum_{i=1}^mx_i^2-\Big(\sum_{i=1}^mx_i\Big)^2}
\end{align}
\subsubsection{一元情形的矩阵形式}
由于 python 中累加形式只能使用\texttt{for}循环进行迭代,运行效率比较低,如果将(1.20)式转化为矩阵形式可以使用\texttt{numpy}等第三方库对其矩阵运算进行加速,公式推导如下:
\begin{align}
  \bar{y}\sum_{i=1}^mx_i\bar{x}\sum_{i=1}^my_i=\sum_{i=1}^m\bar{y}x_i=\sum_{i=1}^m\bar{x}y_i =m\bar{x}\bar{y}=\sum_{i=1}^m\bar{x}\bar{y}
\end{align}

\begin{align}
  \sum_{i=1}^mx_i\bar{x}=\bar{x}\sum_{i=1}^mx_i=\bar{x}\cdot m\cdot \frac{1}{m}\sum_{i=1}^mx_i=m\bar{x}^2=\sum_{i=1}^m\bar{x}^2
\end{align}
将式(1.21)和式(1.22)带入式(1.9)可以得到:
\begin{align}
  w&=\frac{\sum_{i=1}^m(y_ix_i-y_i\bar{x}-x_i\bar{y}+\bar{x}\bar{y})}{\sum_{i=1}^m(x_i^2-\bar{x}x_i-\bar{x}x_i+\bar{x}^2)} \nonumber\\&=\frac{\sum_{i=1}^m(x_i-\bar{x})(y_i-\bar{y})}{\sum_{i=1}^m(x_i-\bar{x})^2}
\end{align}
记$x=(x_1;x_2...x_m)$,$x_d=(x_i-\bar{x};x_2-\bar{x}...x_m-\bar{x})$,$y_d$同$x_d$,再由向量内积的定义可以得到:

\begin{align}
  w=\frac{x_d^Ty_d}{x_d^Tx_d}
\end{align}
\subsubsection{多元}
多元的情形可由一元的情形举一反三推导出,如下:
\begin{align}
  w^*&=\mathop{\arg\min}\limits_{(\hat{w})}\sum_{i=1}^{m}{(y_i-\hat{w}^Tx_i)^2} \nonumber\\ &=\mathop{\arg\min}\limits_{(\hat{w})}\sum_{i=1}^{m}{(y_i-x_i^T\hat{w})^2} \nonumber \\ &=\mathop{\arg\min}\limits_{(\hat{w})}\begin{bmatrix}y_1-x_1^T\hat{w}\dots y_m-x_m^T\hat{w}\end{bmatrix} \cdot \begin{bmatrix}y_1-x_1^T\hat{w} \\ \dots \\ y_m-x_m^T\hat{w}\end{bmatrix}
\end{align}
其中:
\begin{align}
  \begin{bmatrix}y_1-x_1^T\hat{w} \\ y_2-x_2^T\hat{w} \\\dots \\ y_m-x_m^T\hat{w}\end{bmatrix}&=\begin{bmatrix}y_1\\ y_2\\ \dots\\ y_m\end{bmatrix} - \begin{bmatrix}x_1^T\hat{w}\\ x_2^T\hat{w} \\ \dots\\ x_m^T\hat{w}\end{bmatrix} \nonumber \\ &=y-\begin{bmatrix}x_1^T \\ x_2^T \\ \dots \\ x_m^T\end{bmatrix}\cdot \hat{w} \nonumber \\ &= y-X\hat{w} 
\end{align} 
所以:
\begin{align}\hat{w}^*=\mathop{\arg\min}\limits_{(\hat{w})}(y-X\hat{w})^T(y-X\hat{w})\end{align}
现引入两个定理对$\hat{w}$的最优解做进一步推导。

\subsubsection{w预测值的求解}
$Def_{1.28}$:设$D\subset R^n$为非空开凸集$f(x)$是定义在$D$上的实值函数,
且$f(x)$在$D$上二阶连续可微,若$f(x)$的$Hession$矩阵$\nabla^2f(x)$在$D$上半正定则$f(x)$是$D$上的凸函数,
若$\nabla^2f(x$)在$D$上正定则$f(x)$为$D$上的严格凸函数。
$Def_{1.29}$:若$f(x)$为凸函数且$f(x)$一阶连续可微则$x^*$是全局最优解的充要条件为其梯度等于零向量即:
$$\nabla f(x^*)=\textbf{0}$$
\begin{align}E_{\hat{w}}&=(y-X\hat{w})^T(y-X\hat{w})\nonumber \\&= y^Ty-y^TX\hat{w}-\hat{w}^TX^Ty+\hat{w}^TX^TX\hat{w} \end{align}
\begin{align}\frac{\partial E\hat{w}}{\partial \hat{w}}&= 0-X^Ty-X^Ty+(X^TX+X^TX)\hat{w}\nonumber\\&=2X^T(X\hat{w}-y)\end{align}
由式(1.31)再对$\hat{w}^T$求偏导可以得到$E_{\hat{w}} Hession$矩阵的表达式,如下:
\begin{align}
  \nabla^2 E_{\hat{w}}&=\frac{\partial }{\partial w^T}\Big (\frac{\partial E\hat{w}}{\partial w}\Big) \nonumber \\ &=2X^TX 
\end{align}
由$Def_{1.28},Def_{1.29}$当$X^TX$为正定矩阵时,令其梯度为零向量可解得方程的全局最优解即:
\begin{align}
  \hat{w}^*=(X^TX)^{-1}X^Ty
\end{align}
\end{document}